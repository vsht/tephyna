\documentclass[../tephyna.tex]{subfiles}
\begin{document}

\subsection{Verweise}

\mybox{
\begin{tabular}{l}
\ldots \ , wie es in Abb. \# gezeigt ist. \\
\ldots \  ist durch \ldots \  in Abb. \# symbolisiert. \\
Indem man \ldots \ , erhält man das in Abb. \# gezeigte \ldots \
\end{tabular}
}
{
Bsp.: Die Kurve lässt sich mit diesem Ansatz sehr gut nähern, wie es in Abb. 5 gezeigt ist. \\
Bsp.: Die Wechselwirkung beider Teilchen ist durch einen Doppelpfeil in Abb. 2 symbolisiert. \\
Bsp.: Indem man nun zu dem vorher berechneten effektiven Potenzial den Korrekturterm addiert, erhält man den in Abb. 24 gezeigten Verlauf.
}

\mybox{
\begin{tabular}{ l }
\multirow{3}{*}{\minitab[l]{Zu erkennen ist \ldots \  \\ In Abb. \# ist dargestellt, \ldots \  \\ Nebenstehendes Bild zeigt \ldots \  }} \\
\\
\\
\end{tabular}
}
{
Bsp.: Zu erkennen ist das singuläre Verahlten des Feldes im Koordinatenursprung.
}


\mybox{
\begin{tabular}{ l | l  }
\multirow{2}{*}{\minitab[l]{Gemäß \\ Nach }} &
\multirow{2}{*}{\minitab[l]{Abb. \# gilt \ldots \ }} \\
& \\
\end{tabular}
}
{
Bsp.: Nach Abb. 34 gilt: $\tan \alpha = \frac{b}{c}$.
}

\mybox{
\begin{tabular}{ l | l }
\multirow{3}{*}{\minitab[l]{Ein Beispiel hierzu ist in Abb. \# }} &
\multirow{3}{*}{\minitab[l]{dargestellt. \\ gegeben. \\ aufgeführt.}} \\
& \\
& \\
\end{tabular}
}
{
Bsp.: Ein Beispiel hierzu ist in Abb. 18 dargestellt.
}

\mydbox{
\begin{tabular}{l|l|l|l l}
\multirow{4}{*}{\minitab[l]{Weitere Arten}} &
\multirow{4}{*}{\minitab[l]{graphischer \\ zeichnerischer \\ modellhafter}} &
\multirow{4}{*}{\minitab[l]{Darstellung \\ Veranschaulichung}} &
\multirow{4}{*}{\minitab[l]{ dieses Phänomens \\ dieses Sachverhalts \\ dieser Erscheinung \\ \ldots \  \\}} &
\multirow{4}{*}{\minitab[l]{ \Return }} \\
& & & & \\
& & & & \\
& & & & \\
\end{tabular}
}
{
\begin{tabular}{l|l}
\multirow{4}{*}{\minitab[l]{ }} &
\multirow{4}{*}{\minitab[l]{zeigt Abb. \#. \\ werden in Abb \# vorgestellt. \\ sind in Abb. \# dargestellt.}} \\
& \\
& \\
& \\
\end{tabular}
}
{
Bsp.: Weitere Arten modellhafter Darstellung des Welle-Teilchen-Dualismus zeigt Abb. 3.
}

\mybox{

\begin{tabular}{l | l | l }
\multirow{4}{*}{\minitab[l]{\ldots \  macht sich in Abb. \#}} &
\multirow{4}{*}{\minitab[l]{an \ldots \  \\ durch \ldots \  \\ in Form von \ldots \  \\ anhand \ldots \ }} &
\multirow{4}{*}{\minitab[l]{bemerkbar.}} \\
& & \\
& & \\
& & \\
\end{tabular}
}
{
Bsp.: Die Unzulänglichkeiten des numerischen Modells machen sich in Form von starken Oszillationen bemerkbar.
}

\mydbox{
\begin{tabular}{l | l | l  }
\multirow{5}{*}{\minitab[l]{Die in der}} &
\multirow{5}{*}{\minitab[l]{Physik \\ Mathematik \\ Chemie \\ Spektroskopie \\ \ldots \ }} &
\multirow{5}{*}{\minitab[l]{üblichen Bezeichnungen für \ldots \  gehen aus \Return}} \\
  & & \\
  & & \\
  & & \\
  & & \\
\end{tabular}
}
{
\begin{tabular}{l|l | l }
\multirow{3}{*}{\minitab[l]{ }} &
\multirow{3}{*}{\minitab[l]{folgender Abbildung \\ Abb. \# \\ folgender Tabelle}} &
\multirow{3}{*}{\minitab[l]{hervor.}} \\
 &  \\
 &  \\
\end{tabular}
}
{Bsp.: Die in der Kernphysik übliche Bezeichnungen für verschiedene Zerfallsraten gehen aus der folgenden Abbildung hervor.}

\mybox{
\begin{tabular}{l | l | l | l }
\multirow{5}{*}{\minitab[l]{Mit Hilfe von \\ Anhand der \\ Durch den Vergleich mit \\ Aus der \\ Bei genauer Betrachtung von}} &
\multirow{5}{*}{\minitab[l]{Abb. \#}} &
\multirow{5}{*}{\minitab[l]{kann man \\ lässt sich}} &
\multirow{5}{*}{\minitab[l]{ableiten \ldots \  \\ feststellen \ldots \  \\ bestimmen \ldots \  \\ ermitteln \ldots \  \\ erkennen \ldots \ }} \\
& & & \\
& & & \\
& & & \\
& & & \\
\end{tabular}
}
{
Bsp.: Anhand Abb. 4 kann man erkennen, dass der Gleichgewichtszustand nie erreicht wird.
}

\mybox{
\begin{tabular}{l | l }
\multirow{4}{*}{\minitab[l]{Wie man in Abb. \#}} &
\multirow{4}{*}{\minitab[l]{sieht \ldots \  \\ sehen kann \ldots \  \\ erkennt \ldots \  \\ erkennen kann \ldots \ }} \\
& \\
& \\
& \\
\end{tabular}
}
{
Bsp.: Wie man in Abb. 6 sieht, gibt es im Bereich der kleineren Frequenzen erhebliche Unterschiede zwischen den Vorhersagen der Theorie und dem Experiment.
}

\mybox{
\begin{tabular}{l | l | l }
\multirow{4}{*}{\minitab[l]{In Tabelle \#}} &
\multirow{4}{*}{\minitab[l]{sind \ldots \  \\ werden \ldots \ }} &
\multirow{4}{*}{\minitab[l]{aufgelistet. \\ aufgeführt. \\ zusammengestellt. \\ zusammengefasst.}} \\
& & \\
& & \\
& & \\
\end{tabular}
}
{
Bsp.: In Tabelle 3 sind die wichtigsten Formeln noch einmal zusammengestellt.
}

\subsection{Erklärungen}

\mybox{
\begin{tabular}{l}
\ldots \  stellt einen Zusammenhang zwischen \ldots \  und \ldots \  dar. \\
\end{tabular}
}
{
Bsp.: Abb. 16 stellt einen Zusammenhang zwischen der Evolution des Systems und seinem Phasenraumvolumen.
}

\mybox{
\begin{tabular}{l | l | l | l }
\multirow{2}{*}{\minitab[l]{Im}} &
\multirow{2}{*}{\minitab[l]{Diagramm \\ Abb. \#}} &
\multirow{2}{*}{\minitab[l]{ist \\ wird}} &
\multirow{2}{*}{\minitab[l]{\ldots \  gegen \ldots \  aufgetragen}} \\
& & & \\
\end{tabular}
}
{
Bsp.: Im Diagramm ist die Frequenz gegen Photonenenergie aufgetragen.
}

\mybox{
\begin{tabular}{l | l | l }
\multirow{4}{*}{\minitab[l]{Aus dem nebenstehenden Diagramm}} &
\multirow{4}{*}{\minitab[l]{ersieht man, \\ folgt, \\ geht hervor, \\ ist ersichtlich,}} &
\multirow{4}{*}{\minitab[l]{dass \ldots \ }} \\
& & \\
& & \\
& & \\
\end{tabular}
}
{
Bsp.: Aus dem nebenstehenden Diagramm ist ersichtlich, dass der Anstieg bis zum Erreichen der Sättigungsgrenze linear verläuft.
}

\mydbox{
\begin{tabular}{l | l | l | l | l }
\multirow{11}{*}{\minitab[l]{Der}} &
\multirow{11}{*}{\minitab[l]{stetige \\ steile \\ plötzliche \\ lineare \\ leichte \\ deutliche \\ sprunghafte \\ minimale \\ rasche \\ kontinuierliche \\ flache}} &
\multirow{11}{*}{\minitab[l]{Anstieg \\ Verlauf}} &
\multirow{11}{*}{\minitab[l]{der Kurve }} &
\multirow{11}{*}{\minitab[l]{(im Bereich \ldots \ ) \Return \\ (für \ldots \ )}} \\
& & & & \\
& & & & \\
&  & & & \\
&  & & & \\
&  & & & \\
&  & & & \\
&  & & & \\
&  & & & \\
&  & & & \\
&  & & & \\
\end{tabular}
}
{
\begin{tabular}{l|l}
\multirow{2}{*}{\minitab[l]{}} &
\multirow{2}{*}{\minitab[l]{ist auf \ldots \  zurückzuführen. \\ bedeutet, dass \ldots \ }} \\
 &  \\
\end{tabular}
}
{
Bsp.: Der flache Anstieg der Kurve ist auf starke Dämpfung zurückzuführen.
}

\mybox{
\begin{tabular}{l | l | l }
\multirow{2}{*}{\minitab[l]{Abb. \#}} &
\multirow{2}{*}{\minitab[l]{zeigt \\ verdeutlicht}} &
\multirow{2}{*}{\minitab[l]{das Zustandekommen \ldots \ }} \\
& & \\
\end{tabular}
}
{
Bsp.: Abb. 18 verdeutlicht das Zustandekommen der Interferenzfigur infolge der Beugung am Doppelspalt.
}

\mybox{
\begin{tabular}{l | l | l | l }
\multirow{3}{*}{\minitab[l]{In Abb. \#}} &
\multirow{3}{*}{\minitab[l]{\ldots \ }} &
\multirow{3}{*}{\minitab[l]{aus Gründen der Übersichtlichkeit \\ zur Verdeutlichung \\ der Deutlichkeit halber}} &
\multirow{3}{*}{\minitab[l]{nur schematisch angedeutet. \\ stark vereinfacht dargestellt. \\ übertrieben groß gezeichnet.}} \\
& & & \\
& & & \\
\end{tabular}
}
{
Bsp.: In Abb. 5 sind die Winkel aus Gründen der Übersichtlichkeit übertrieben groß gezeichnet.
}

\mybox{
\begin{tabular}{l | l | l }
\multirow{3}{*}{\minitab[l]{Hier \\ An dieser Stelle \\ In diesem Bild}} &
\multirow{3}{*}{\minitab[l]{ist \\ sind}} &
\multirow{3}{*}{\minitab[l]{zur Verdeutlichung zusätzlich \ldots \ }} \\
& & \\
& & \\
\end{tabular}
}
{
Bsp.: Hier sind zur Verdeutlichung zusätzlich die zugehörigen Stoßprozesse eingezeichnet.
}

\mydbox{
\begin{tabular}{l | l | l }
\multirow{3}{*}{\minitab[l]{Folgende(r/s) \\ Der/die/das charakteristische \\ Der/die/das entsprechende}} &
\multirow{3}{*}{\minitab[l]{Graph \\ Kurve \\ Diagramm}} &
\multirow{3}{*}{\minitab[l]{ergibt sich, wenn man \Return}} \\
 & & \\
 & & \\
\end{tabular}
}
{
\begin{tabular}{l | l}
\multirow{3}{*}{\minitab[l]{  }} &
\multirow{3}{*}{\minitab[l]{\ldots \  in Abhängigkeit von \ldots \  darstellt. \\ \ldots \  gegen \ldots \  aufträgt.}} \\
& \\
& \\
\end{tabular}
}
{
Bsp.: Das charakteristische Diagramm ergibt sich, wenn man die Intensität in Abhängigkeit vom Winkel $\alpha$ darstellt.
}

\mybox{
\begin{tabular}{l | l  }
\multirow{4}{*}{\minitab[l]{Aus Symmetriegründen}} &
\multirow{4}{*}{\minitab[l]{genügt es, \ldots \  \\ beschränkt man sich auf \ldots \  \\ ist es ausreichend, \ldots \  \\ reicht es, \ldots \ }} \\
& \\
& \\
& \\
\end{tabular}
}
{
Bsp.: Aus Symmetriegründen ist es ausreichend, nur den Fall $x>0$ zu betrachten.
}

\end{document}
