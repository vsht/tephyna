\documentclass[../tephyna.tex]{subfiles}
\begin{document}

\mybox{
\begin{tabular}{l}
Berücksichtigt man nun, dass \ldots \ \\
Daraus erkennt man \ldots \  \\
Es zeigt sich, dass \ldots \  \\
Dies erklärt, warum \ldots \  \\
Dies zeigt sich insbesondere bei \ldots \  \\
Wie man sieht, \ldots \  \\
Aus lässt sich auf \ldots \  schließen. \\
Das Gleiche gilt auch für \ldots \  \\
\ldots \  erfordert Kenntnisse aus \ldots \  \\
\ldots \  was auch aus \ldots \  hervorgeht. \\
Im Allgemeinen ist es nicht möglich, \ldots \  \\
Dies äußert sich in \ldots \  \\
Um feststellen zu können, ob \ldots \  \\
\end{tabular}
}
{
Bsp.: Zur Deutung dieses Phänomens bedarf es einer neuen Quantentheorie jenseits des Standardmodells.
}

\mybox{
\begin{tabular}{l | l | l }
\multirow{3}{*}{\minitab[l]{\ldots \  liegt}} &
\multirow{3}{*}{\minitab[l]{teilweise \\ größtenteils \\ unter anderem}} &
\multirow{3}{*}{\minitab[l]{daran, dass \ldots \ }} \\
& & \\
& & \\
\end{tabular}
}
{
Bsp.: Das liegt unter anderem daran, dass in dieser einfachen Theorie die Selbstwechselwirkung der Teilchen nicht berücksichtigt wird.
}

\mybox{
\begin{tabular}{l | l | l }
\multirow{3}{*}{\minitab[l]{Hieraus \\ Daraus}} &
\multirow{3}{*}{\minitab[l]{geht hervor, \\ folgt, \\ kann geschlossen werden,}} &
\multirow{3}{*}{\minitab[l]{dass \ldots \ }} \\
& & \\
& & \\
\end{tabular}
}
{
Bsp.: Hieraus folgt, dass es keine longitudinalen elektromagnetischen Wellen gibt.
}

\mybox{
\begin{tabular}{l | l | l }
\multirow{2}{*}{\minitab[l]{Es lässt sich \\ Man kann}} &
\multirow{2}{*}{\minitab[l]{beweisen, \\ zeigen,}} &
\multirow{2}{*}{\minitab[l]{dass \ldots \ }} \\
& & \\
\end{tabular}
}
{
Bsp.: Man kann zeigen, dass es nicht mehr als 8 solche Zustände geben kann.
}

\mybox{
\begin{tabular}{l | l }
\multirow{5}{*}{\minitab[l]{\ldots \ }} &
\multirow{5}{*}{\minitab[l]{deutet darauf hin \ldots \  \\ bedingt \ldots \  \\ hat zur Folge \ldots \  \\ impliziert \ldots \ }} \\
& \\
& \\
& \\
& \\
\end{tabular}
}
{
Bsp.: Diese Phasenverschiebung hat zur Folge, dass sich die Wellen konstruktiv überlagern können.
}

\mybox{
\begin{tabular}{l | l | l }
\multirow{2}{*}{\minitab[l]{Nach}} &
\multirow{2}{*}{\minitab[l]{heutiger \\ damaliger}} &
\multirow{2}{*}{\minitab[l]{Auffassung \ldots}} \\
& & \\
\end{tabular}
}
{
Bsp.: Nach heutiger Auffassung gibt es keine absolute Zeit.
}

\mybox{
\begin{tabular}{l | l | l | l}
\multirow{2}{*}{\minitab[l]{\ldots \ }} &
\multirow{2}{*}{\minitab[l]{lässt sich \\ kann man sich}} &
\multirow{2}{*}{\minitab[l]{folgendermaßen}} &
\multirow{2}{*}{\minitab[l]{erklären. \\ klar machen.}} \\
& & \\
\end{tabular}
}
{
Bsp.: Das Auftreten der Resonanz kann man sich hier folgendermaßen erklären.
}

\mybox{
\begin{tabular}{l | l | l | l | l}
\multirow{5}{*}{\minitab[l]{Diese Überlegung \\ Dieses Modell \\ Diese Theorie}} &
\multirow{5}{*}{\minitab[l]{kann man \\ lässt sich}} &
\multirow{5}{*}{\minitab[l]{auch \\ unter Umständen auch \\ interessanterweise auch \\ glücklicherweise auch \\leider nicht }} &
\multirow{5}{*}{\minitab[l]{auf \ldots \ }} &
\multirow{5}{*}{\minitab[l]{erweitern \\ ausweiten \\ übertragen \\ anwenden}} \\
& & & & \\
& & & & \\
& & & & \\
& & & & \\
\end{tabular}
}
{
Bsp.: Dieses Modell lässt sich interessanterweise auch auf Mehrteilchensysteme erweitern.
}

\mybox{
\begin{tabular}{l | l | l }
\multirow{6}{*}{\minitab[l]{Wobei \ldots \  noch}} &
\multirow{6}{*}{\minitab[l]{überdacht \\ präzisiert \\ geklärt \\ untersucht \\ erforscht \\ verbessert}} &
\multirow{6}{*}{\minitab[l]{werden muss.}} \\
& & \\
& & \\
& & \\
& & \\
& & \\
\end{tabular}
}
{
Bsp.: Wobei der Zusammenhang zwischen den beiden Phänomenen noch geklärt werden muss.
}

\mybox{
\begin{tabular}{l | l | l | l | l}
\multirow{3}{*}{\minitab[l]{Der Sachverhalt, \\ Die Tatsache,}} &
\multirow{3}{*}{\minitab[l]{dass \ldots \  kommt \ldots \ }} &
\multirow{3}{*}{\minitab[l]{explizit \\ deutlich \\ eindeutig}} &
\multirow{3}{*}{\minitab[l]{zum}} &
\multirow{3}{*}{\minitab[l]{Ausdruck \ldots \  \\ Vorschein \ldots \ }} \\
& & & & \\
& & & & \\
\end{tabular}

}
{
Bsp.: Die Tatsache, dass sich das Lösen dieser Gleichung auf das Lösen einer linearen DGL reduzieren lässt, kommt explizit zum Vorschein, sobald man die entsprechende Variablentransformation vornimmt.
}

\mybox{
\begin{tabular}{l | l | l | l}
\multirow{4}{*}{\minitab[l]{Den Ausgangspunkt \\ Die Voraussetzung \\ Die Planung}} &
\multirow{4}{*}{\minitab[l]{für diese(s)}} &
\multirow{4}{*}{\minitab[l]{Vorhaben \\ Forschungsprojekt \\ Untersuchung}} &
\multirow{4}{*}{\minitab[l]{bildet(e) \ldots \ \\ setzt voraus \ldots \ \\ ist \ldots \ \\ war \ldots \ }} \\
& & & \\
& & & \\
& & & \\
\end{tabular}
}
{
Bsp.: Den Ausgangspunkt für dieses Forschungsprojekt bildete der Wunsch, die Physik der ultrakalten Gase besser verstehen zu können.
}

\mybox{
\begin{tabular}{ l | l }
\multirow{4}{*}{\minitab[l]{Um \ldots \  zu \ldots \ , ist es }} &
\multirow{4}{*}{\minitab[l]{notwendig \ldots \\ wichtig \ldots  \\ unabdingbar \ldots  \\ entscheidend \ldots }} \\
& \\
& \\
& \\
\end{tabular}
}
{
Bsp.: Um diesen Effekt beobachten zu können, ist es unabdingbar, mit äußerster Präzision vorzugehen.
}

\mydbox{
\begin{tabular}{l | l | l | l | l }
\multirow{5}{*}{\minitab[l]{Die Einführung}} &
\multirow{5}{*}{\minitab[l]{einer \\ der}} &
\multirow{5}{*}{\minitab[l]{neuen \\ dimensionslosen}} &
\multirow{5}{*}{\minitab[l]{Größe \\ Konstante \\ Variable}} &
\multirow{5}{*}{\minitab[l]{erweist sich}} \\
 & & & & \\
 & & & & \\
 & & & & \\
 & & & & \\
\end{tabular}
}
{
\begin{tabular}{l | l | l | l | l }
\multirow{5}{*}{\minitab[l]{ }} &
\multirow{5}{*}{\minitab[l]{bei der Untersuchung \\ bei der Erforschung \\ bei der Betrachtung \\ für das Verständnis }} &
\multirow{5}{*}{\minitab[l]{von \ldots \  als}} &
\multirow{5}{*}{\minitab[l]{zweckmäßig, \\ sinnvoll, \\ gerechtfertigt, \\ schlüssig, \\folgerichtig, }} &
\multirow{5}{*}{\minitab[l]{da \ldots \ }} \\
& & & & \\
& & & & \\
& & & & \\
& & & & \\
\end{tabular}
}
{
Bsp.: Die Einführung einer dimensionslosen Größe erweist sich bei der Betrachtung dieses Problems als zweckmäßig, da die Bewegungsgleichungen dadurch viel einfacher werden.
}

\mybox{
\begin{tabular}{l | l | l | l | l}
\multirow{6}{*}{\minitab[l]{\ldots \  erscheint \Return \\zunächst}} &
\multirow{6}{*}{\minitab[l]{unmöglich, \\ widersprüchlich, \\ paradox, \\ unlogisch, \\ verwunderlich, \\ widersinnig,}} &
\multirow{6}{*}{\minitab[l]{lässt sich \Return \\ aber \ldots \ }} &
\multirow{6}{*}{\minitab[l]{leicht \\ schlüssing \\ plausibel \\ logisch \\folgendermaßen}} &
\multirow{6}{*}{\minitab[l]{erklären. \\ verstehen. \\ deuten.}} \\
& & & & \\
& & & & \\
& & & & \\
& & & & \\
& & & & \\
\end{tabular}
}
{
Bsp.: Der plötzliche Temperaturanstieg erscheint zunächst paradox, lässt sich aber mit dem 1. Hauptsatz der Thermodynamik plausibel erklären.
}

\mybox{
\begin{tabular}{l | l | l }
\multirow{5}{*}{\minitab[l]{\ldots \  ist in guter Näherung als}} &
\multirow{5}{*}{\minitab[l]{konstant \\ stationär \\ sinusförmig \\ linear \\ \ldots \ }} &
\multirow{5}{*}{\minitab[l]{anzusehen. \\ zu betrachten. \\ aufzufassen. \\ beschreibbar.}} \\
& & \\
& & \\
& & \\
& & \\
\end{tabular}
}
{
Bsp.: Die Kurve ist in guter Näherung als sinusförmig zu betrachten.
}

\mybox{
\begin{tabular}{l | l | l }
\multirow{4}{*}{\minitab[l]{Wäre dies}} &
\multirow{4}{*}{\minitab[l]{anders, \\ (nicht) der Fall, \\ falsch, \\ wahr,}} &
\multirow{4}{*}{\minitab[l]{so müsste \ldots \ }} \\
& & \\
& & \\
& & \\
\end{tabular}
}
{
Bsp.: Wäre dies nicht der Fall, so wäre hier eine solche Näherung unmöglich.
}

\mybox{
\begin{tabular}{l | l | l |l}
\multirow{5}{*}{\minitab[l]{Noch}} &
\multirow{5}{*}{\minitab[l]{deutlicher \\ eleganter \\ einfacher \\ komplizierter \\ problematischer}} &
\multirow{5}{*}{\minitab[l]{wird \\ werden}} &
\multirow{5}{*}{\minitab[l]{\ldots \ ,  wenn \ldots \ }} \\
& & & \\
& & & \\
& & & \\
& & & \\
\end{tabular}
}
{
Bsp.: Noch eleganter werden diese Gleichungen, wenn man die Integrale gleich auswertet.
}

\mybox{
\begin{tabular}{ l | l }
\multirow{5}{*}{\minitab[l]{Ein Vergleich mit \ldots \ }} &
\multirow{5}{*}{\minitab[l]{zeigt \ldots \ \\ ergibt \ldots \  \\ beweist \ldots \  \\ belegt \ldots \  \\ rechtfertigt \ldots \ }} \\
& \\
& \\
& \\
& \\
\end{tabular}
}
{
Bsp.: Ein Vergleich mit dem experimentell ermittelten Wert rechtfertigt die zuvor gemachten Näherungen.
}

\mybox{

\begin{tabular}{l | l | l | l}
\multirow{2}{*}{\minitab[l]{Für \ldots \ }} &
\multirow{2}{*}{\minitab[l]{kann man \\ lassen sich}} &
\multirow{2}{*}{\minitab[l]{folgende \\ entsprechende}} &
\multirow{2}{*}{\minitab[l]{Überlegungen anstellen.}} \\
& & & \\
\end{tabular}
}
{
Bsp.: Für das Auftreten von Schwebungen lassen sich folgende Überlegungen anstellen.
}

\mybox{
\begin{tabular}{l | l | l }
\multirow{8}{*}{\minitab[l]{\ldots \  ist ein}} &
\multirow{8}{*}{\minitab[l]{weiteres \\ klares \\ (besonders) starkes \\ eindeutiges \\ sicheres \\ deutliches \\ zusätzliches \\ mögliches}} &
\multirow{8}{*}{\minitab[l]{Indiz für \ldots \ }} \\
& & \\
& & \\
& & \\
& & \\
& & \\
& & \\
& & \\
\end{tabular}
}
{
Bsp.: Dieses Leuchten ist ein besonders starkes Indiz für die Zündprozesse innerhalb der Röhre.
}

\mybox{
\begin{tabular}{l | l | l }
\multirow{2}{*}{\minitab[l]{\ldots \  rührt}} &
\multirow{2}{*}{\minitab[l]{zum großen Teil \\ hauptsächlich}} &
\multirow{2}{*}{\minitab[l]{von \ldots \  her.}} \\
& & \\
\end{tabular}
}
{
Bsp.: Die beobachtete Instabilität rührt zum großen Teil von der Coulomb-Abstoßung her.
}

\mybox{
\begin{tabular}{l | l | l }
\multirow{4}{*}{\minitab[l]{\ldots \  eignet sich}} &
\multirow{4}{*}{\minitab[l]{somit \\ zudem \\ deshalb \\ daher}} &
\multirow{4}{*}{\minitab[l]{als Maß für \ldots \ }} \\
& & \\
& & \\
& & \\
\end{tabular}
}
{
Bsp.: Die gemessene Spannung eignet sich somit als Maß für die Intensität.
}

\mybox{
\begin{tabular}{l | l | l | l }
\multirow{2}{*}{\minitab[l]{\ldots \ da \\ \ldots \ weil}} &
\multirow{2}{*}{\minitab[l]{es}} &
\multirow{2}{*}{\minitab[l]{in diesem \\ im letzteren}} &
\multirow{2}{*}{\minitab[l]{Fall zu \ldots \  kommt.}} \\
& & & \\
\end{tabular}
}
{
Bsp.:Hochfrequente Schwingungen sind hier aber nicht erwünscht, weil es in diesem Fall zu starkem Rauschen kommt.
}

\mybox{
\begin{tabular}{l | l | l }
\multirow{5}{*}{\minitab[l]{Als (eine) Größe, die \ldots \ }} &
\multirow{5}{*}{\minitab[l]{kennzeichnet, \\ beschreibt, \\ angibt, \\vorgibt, \\bestimmt,}} &
\multirow{5}{*}{\minitab[l]{verwendet man \ldots \  \\ dient \ldots \ }} \\
& & \\
& & \\
& & \\
& & \\
\end{tabular}
}
{
Bsp.: Als Größen, welche die Dynamik eines Teilchens bestimmen, dienen sein Ort und sein Impuls.
}

\mybox{
\begin{tabular}{ l | l }
\multirow{3}{*}{\minitab[l]{Diese Tatsache \\ Dieser Grundsatz \\ Diese Beobachtung}} &
\multirow{3}{*}{\minitab[l]{ist (nur) zu verstehen, wenn \ldots \ }} \\
& \\
& \\
\end{tabular}
}
{
Bsp.: Diese Beobachtung ist nur zu verstehen, wenn man den gesamten Prozess einer quantenmechanischen Behandlung unterzieht.
}

\end{document}
