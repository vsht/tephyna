\documentclass[../tephyna.tex]{subfiles}
\begin{document}
\mybox{
\begin{tabular}{l}
ergänzend = komplementär \\
a priori = von vornherein \\
a posteriori = im Nachhinein \\
irrelevant, nicht relevant = unwichtig \\
evident = offensichtlich\\
deduktiv = aus theoretischen Überlegungen hergeleitet \\
empirisch = aus experimentellen Beobachtungen \\
detailliert = ausführlich \\
sich um \ldots \ differieren = sich um \ldots \ unterscheiden\\
am Anfang = eingangs, zuerst, anfangs \\
vorhergehend, vorangehend, vorausgehend, vorherig $\leftrightarrow$ nachfolgend, nächstfolgend, \\
\end{tabular}
}
{
}

\mybox{
\begin{tabular}{l | l | l | l}
\multirow{4}{*}{Das Ziel}  &  \multirow{4}{*}{\minitab[l]{diese(r/s) \\ de(r/s) vorliegender(r/s)} } &
\multirow{4}{*}{\minitab[l]{Arbeit \\ Artikels \\ Berichts \\ Kapitels}} & \multirow{4}{*}{ist es, \ldots \ } \\
 & & & \\
& & & \\
& & & \\
\end{tabular}
}
{
Bsp.: Ziel dieser Arbeit ist es, einen Einblick in die moderne Festkörperphysik zu geben und ihre quantentheoretischen Grundlagen zu erläutern.
}

\mybox{
\begin{tabular}{l | l | l }
\multirow{3}{*}{Historisch entsprang}  &
\multirow{3}{*}{\minitab[l]{die Entwicklung \ldots \ \\ die Erfindung \ldots \ \\ die Entstehung \ldots \ } } &
\multirow{3}{*}{\minitab[l]{dem Wunsch, \ldots \ }} \\
& & \\
& & \\
\end{tabular}
}
{
Bsp.: Historisch entsprang die Entwicklung des Zyklontrons dem Wunsch, noch höhere Energien zu erreichen.
}

\mybox{
\begin{tabular}{l | l | l | l }
\multirow{4}{*}{\ldots \ gab(en)}  &  \multirow{4}{*}{\minitab[l]{den Anstoß\\ den Auftakt \\ die Veranlassung \\ den Anlass} } &
\multirow{4}{*}{zur} &
\multirow{4}{*}{\minitab[l]{Entwicklung \ldots \ \\ Gründung \ldots \ \\ Entstehung \ldots \ \\ Erforschung \ldots \ }} \\
& & & \\
& & & \\
& & & \\
\end{tabular}
}
{
Bsp.: Die Quantenhypothese von Max Planck gab schließlich den Anstoß zur Entwicklung der Quantenmechanik, die unser physikalisches Weltbild für immer veränderte.
}

\mybox{
\begin{tabular}{l | l | l | l | l}
\multirow{5}{*}{\minitab[l]{Dieses Kapitel \\ Diese Einführung \\ Dieser Abschnitt \\ Dieser Bericht \\ Dieses Buch}}  &
\multirow{5}{*}{soll} &
\multirow{5}{*}{\minitab[l]{den Leser \\ Sie \\ andere \Return \\ Wissenschaftler \\ alle Interessenten}} &
\multirow{5}{*}{\minitab[l]{mit der Theorie \ldots \ \\ mit der Funktionsweise \ldots \ \\ mit den Grundlagen \ldots \ }} &
 \multirow{5}{*}{vertraut machen.} \\
& & & & \\
& & & & \\
& & & & \\
& & & & \\
\end{tabular}
}
{Bsp.: Diese Einführung soll Sie mit der Funktionsweise des neuen Massenspektrographen XYZ vertraut machen und Ihnen zugleich die Grundzüge der Massensprektroskopie erläutern.}

\mybox{
\begin{tabular}{ l | l }
\multirow{3}{*}{\minitab[l]{\ldots \ soll zum besseren Verständnis \ldots \ }} &
\multirow{3}{*}{\minitab[l]{verhelfen. \\ beitragen. \\ dienen.}} \\
& \\
& \\
\end{tabular}
}
{Bsp.: Diese Arbeit soll zum besseren Verständnis der Entstehung von Gravitationswellen beitragen.}

\mybox{
\begin{tabular}{l | l | l }
\multirow{5}{*}{\minitab[l]{Die Vorliegende}} &
\multirow{5}{*}{\minitab[l]{Arbeit \\ Untersuchung \\ Abhandlung \\ Übersicht \\ Studie}} &
\multirow{5}{*}{\minitab[l]{befasst sich mit \ldots \ \\  widmet sich der Frage, \ldots \ \\ hat die Untersuchung \ldots \ zum Ziel.}} \\
& & \\
& & \\
& & \\
& & \\
\end{tabular}
}
{
Bsp.: Die Vorliegende Arbeit befasst sich mit der störungstheoretischen Berechnung von Grundzustandsenergien in extremen Nichtgleichgewichtszuständen.
}

\end{document}
