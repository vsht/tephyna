\documentclass[../tephyna.tex]{subfiles}
\begin{document}

\subsection{Angaben}

\mybox{
\begin{tabular}{l}
Für \ldots \  erhält man \ldots \  \\
Das ergibt \ldots \  \\
Die Einheit \ldots \  ist (demnach) das \ldots \  \\
\ldots \  wird generell mit \ldots \  symbolisiert und \ldots \  genannt \\
\end{tabular}
}
{
Bsp.: Diese Summe wird generell mit $\alpha$ symbolisiert und Madelung-Konstante genannt.
}

\mybox{
\begin{tabular}{ l | l }
\multirow{5}{*}{\minitab[l]{Für \ldots \  \\ Aus den Messungen \\ Es \\ Bei Messungen \\ Dabei}} &
\multirow{5}{*}{\minitab[l]{ergeben sich folgende Werte \ldots \ }} \\
& \\
& \\
& \\
& \\
\end{tabular}
}
{
Bsp.: Aus den Messungen ergeben sich folgende Werte für die Leitfähigkeit.
}

\mybox{
\begin{tabular}{l | l | l }
\multirow{5}{*}{\minitab[l]{\ldots \  liefert \ldots \  , was mit dem}} &
\multirow{5}{*}{\minitab[l]{beobachteten \\ gemessenen \\ zu erwartenden}} &
\multirow{5}{*}{\minitab[l]{Wert von \ldots \   sehr gut übereinstimmt.}} \\
& & \\
& & \\
& & \\
& & \\
\end{tabular}
}
{
Bsp.: Die Rechnung liefert eine Stromstärke in Höhe von 120\,mA, was mit dem gemessenen Wert sehr gut übereinstimmt.
}

\mybox{
\begin{tabular}{l | l}
\multirow{4}{*}{\minitab[l]{Der Wert  \ldots \  \\ Der Zahlenwert \ldots \  \\ Die Konstante \ldots \ }} &
\multirow{4}{*}{\minitab[l]{beschreibt \ldots \  \\ bestimmt \ldots \  \\ errechnet sich aus \ldots \  \\ kennzeichnet \ldots \ }} \\
& \\
& \\
& \\
\end{tabular}
}
{
Bsp.: Die Konstante $C$ errechnet sich aus der Integration über alle Raumrichtungen.
}

\subsection{Zusammenhänge und Abhängigkeiten}

\mybox{
\begin{tabular}{l}
\ldots \  hängt mit \ldots \  folgendermaßen zusammen \ldots \ \\
\ldots \  ist mit \ldots \  folgendermaßen verknüpft \ldots \  \\
Der Zusammenhang zwischen \ldots \  und \ldots \  wird durch \ldots \  gegeben. \\
Aus diesen Gesetzen lässt sich die Beziehung herleiten, \ldots \  \\
\ldots \  ist ein Maß für \ldots \  \\
Diese Gleichug beschreibt \ldots \
\end{tabular}
}
{
Bsp.: Der Zusammenhang zwischen Temperatur und Leitfähigkeit wird durch Gl. 7 gegeben.
}

\mybox{
\begin{tabular}{l | l | l }
\multirow{3}{*}{\minitab[l]{\ldots \  kann man \\ \ldots \  lässt sich}} &
\multirow{3}{*}{\minitab[l]{folgende(r/n) Zusammenhang zwischen \ldots \  und \ldots \ }} &
\multirow{3}{*}{\minitab[l]{herleiten. \\ angeben. \\ ermitteln.}} \\
& & \\
& & \\
\end{tabular}
}
{
Bsp.: Durch Umformen der Gleichung 5 kann man folgenden Zusammenhang zwischen der Stromstärke und der Teilchenzahl herleiten.
}

\mybox{
\begin{tabular}{l | l | l }
\multirow{3}{*}{\minitab[l]{Im Folgenden soll nun untersucht werden}} &
\multirow{3}{*}{\minitab[l]{wie \\ inwieweit \\ auf welche Art und Weise}} &
\multirow{3}{*}{\minitab[l]{\ldots \  von \ldots \  abhängt.}} \\
& & \\
& & \\
\end{tabular}
}
{
Bsp.: Im Folgenden soll nun untersucht werden, wie der differentielle Stoßquerschnitt von der Geometrie molekularer Stoßpartner abhängt.
}

\mybox{
\begin{tabular}{ l | l }
\multirow{5}{*}{\minitab[l]{\ldots \  ist gegeben durch}} &
\multirow{5}{*}{\minitab[l]{die Summe \ldots \  \\ die Differenz \ldots \  \\ das Produkt \ldots \  \\ den Quotienten \ldots \  \\ die Verknüpfung \ldots \ }} \\
& \\
& \\
& \\
& \\
\end{tabular}
}
{
Bsp.: Die Gesamtwellenfunktion ist gegeben durch ein Tensorprodukt aus dem Spin- und dem Ortsanteil.
}

\mybox{
\begin{tabular}{ l | l }
\multirow{3}{*}{\minitab[l]{\ldots \  unterliegt \\ \ldots \  folgt \\ \ldots \  gehorcht}} &
\multirow{3}{*}{\minitab[l]{einer Gesetzmäßigkeit \ldots \  \\ einem Exponentialgesetz \ldots \  \\ einer Verteilung \ldots \ }} \\
& \\
& \\
\end{tabular}
}
{
Bsp.: Der radioaktive Zerfall unterliegt einem Exponentialgesetz.
}

\mybox{
\begin{tabular}{l | l | l | l | l}
\multirow{3}{*}{\minitab[l]{Aus}} &
\multirow{3}{*}{\minitab[l]{dieser Beziehung \\ dem obigen \\ diesen Beispielen}} &
\multirow{3}{*}{\minitab[l]{ist}} &
\multirow{3}{*}{\minitab[l]{zu ersehen, \\ ersichtlich,}} &
\multirow{3}{*}{\minitab[l]{dass \ldots \ }} \\
& & & & \\
& & & & \\
\end{tabular}
}
{
Bsp.: Aus diesen Beispielen ist zu ersehen, dass die vorliegende Theorie viele Fragestellungen nur unbefriedigend beantworten kann.
}

\subsection{Rechnungen}
\mybox{
\begin{tabular}{l | l | l | l | l }
\multirow{7}{*}{\minitab[l]{\ldots \  lasst sich mit}} &
\multirow{7}{*}{\minitab[l]{der \\ dieser \\ der obigen}} &
\multirow{7}{*}{\minitab[l]{Beziehung}} &
\multirow{7}{*}{\minitab[l]{angenähert \\ in guter Näherung \\ exakt \\ in erster Näherung \\ größenordnungsmäßig \\ in nullter Näherung}} &
\multirow{7}{*}{\minitab[l]{berechnen. \\ bestimmen. \\ ermitteln. \\ errechnen. \\ angeben.}} \\
& & & \\
& & & \\
& & & \\
& & & \\
& & & \\
& & & \\
\end{tabular}
}
{
Bsp.: Die entsprechende Frequenz lässt sich mit der obigen Beziehung angenähert angeben.
}

\mybox{
\begin{tabular}{l | l | l | l }
\multirow{4}{*}{\minitab[l]{Bei}} &
\multirow{4}{*}{\minitab[l]{allen \\ diesen \\ nachfolgenden}} &
\multirow{4}{*}{\minitab[l]{Berechnungen ist}} &
\multirow{4}{*}{\minitab[l]{es sinnvoll, \ldots \  \\ vorausgesetzt, \ldots \  \\ anzunehmen \ldots \  \\ zu beachten \ldots \ }} \\
& & & \\
& & & \\
& & & \\
\end{tabular}
}
{
Bsp.: Bei allen Berechnungen ist zu beachten, dass der Einfluss der Gravitation völlig vernachlässigt wird.
}

\mydbox{
\begin{tabular}{l | l | l | l | l }
\multirow{4}{*}{\minitab[l]{Durch die}} &
\multirow{4}{*}{\minitab[l]{Verkleinerung \\ Vergrößerung \\ Erhöhung \\ Erniedrigung}} &
\multirow{4}{*}{\minitab[l]{von \ldots \  kann so eine}} &
\multirow{4}{*}{\minitab[l]{beliebig \\ sehr}} &
\multirow{4}{*}{\minitab[l]{genaue \Return}} \\
 & & & & \\
 & & & & \\
 & & & & \\
\end{tabular}
}
{
\begin{tabular}{l | l | l }
\multirow{4}{*}{\minitab[l]{ }} &
\multirow{4}{*}{\minitab[l]{Berechnung \\ Bestimmung}} &
\multirow{4}{*}{\minitab[l]{von \ldots \  erfolgen.}} \\
& & \\
& & \\
& & \\
\end{tabular}
}
{
Bsp.: Durch die Erniedrigung der Temperatur auf unter 10\,K kann so eine sehr genaue Bestimmung der Konstanten d erfolgen.
}

\mybox{
\begin{tabular}{l | l | l | l }
\multirow{2}{*}{\minitab[l]{Wenn \ldots \  so groß}} &
\multirow{2}{*}{\minitab[l]{wird, \\ werden}} &
\multirow{2}{*}{\minitab[l]{dass}} &
\multirow{2}{*}{\minitab[l]{\ldots \  in die Nähe \ldots \  kommt \ldots \ \\ \ldots \  mit \ldots \  vergleichbar wird \ldots \ }} \\
& & & \\
\end{tabular}
}
{
Bsp.: Wenn die Massen so groß werden, dass die Stärke der gravitativen Wechselwirkung mit anderen Wechselwirkungen vergleichbar wird, lassen sich völlig neue Phänomene beobachten.
}

\mybox{
\begin{tabular}{l | l | l | l | l}
\multirow{3}{*}{\minitab[l]{Aus \ldots \  kann \Return \\ \ldots \  mit Hilfe}} &
\multirow{3}{*}{\minitab[l]{des \\ eines speziellen \\ eines vorgegebenen}} &
\multirow{3}{*}{\minitab[l]{Umrechnungsfaktors}} &
\multirow{3}{*}{\minitab[l]{berechnet \\ errechnet \\ bestimmt}} &
\multirow{3}{*}{\minitab[l]{werden.}} \\
& & & & \\
& & & & \\
\end{tabular}
}
{
Bsp.: Aus der Umlaufzeit kann mit Hilfe eines speziellen Umrechnungsfaktors die Energie berechnet werden.
}

\subsection{Herleitungen}

\mybox{
\begin{tabular}{l}
Für \ldots \  lässt sich schreiben \ldots \\
Führt man \ldots \  eine Proportionalitätskonstante ein,\ldots \  \\
Das Minuszeichen berücksichtigt \ldots \  \\
Unnötige Komplikationen bei späteren Herleitungen lassen sich vermeiden, wenn \ldots \  \\
\end{tabular}
}
{
Bsp.: Das Minuszeichen berücksichtigt die Richtung der Rückstellkraft.
}

\mybox{
\begin{tabular}{l | l | l | l }
\multirow{4}{*}{\minitab[l]{Die Formel \ldots \  \\ Die Gleichung \ldots \  \\ Die Beziehung \ldots \ }} &
\multirow{4}{*}{\minitab[l]{kann}} &
\multirow{4}{*}{\minitab[l]{wie folgt \ldots \  \\ folgendermaßen \ldots \  \\ unter der Annahme \ldots \  \\ mit Hilfe \ldots \ }} &
\multirow{4}{*}{\minitab[l]{hergeleitet werden.}} \\
& & & \\
& & & \\
& & & \\
\end{tabular}
}
{
Bsp.: Diese Beziehung kann wie folgt aus der Dirac-Gleichung hergeleitet werden.
}

\mybox{
\begin{tabular}{l | l | l }
\multirow{4}{*}{\minitab[l]{Hieraus folgt die}} &
\multirow{4}{*}{\minitab[l]{obige \\ explizite \\ empirische \\ für die Praxis wichtige}} &
\multirow{4}{*}{\minitab[l]{Formel \ldots \  \\ \ldots \ }} \\
& & \\
& & \\
& & \\
\end{tabular}
}
{
Bsp.: Hieraus folgt die für die Praxis wichtige Linsenmacherformel.
}

\mydbox{
\begin{tabular}{l | l | l | l }
\multirow{3}{*}{\minitab[l]{Kombiniert man die \ldots \ }} &
\multirow{3}{*}{\minitab[l]{erarbeiteten \\ angegebenen \\hergeleiteten}} &
\multirow{3}{*}{\minitab[l]{Ergebnisse, \\ Formeln, \\ Beziehungen,}} &
\multirow{3}{*}{\minitab[l]{so}} \\
 & & & \\
 & & & \\
\end{tabular}
}
{
\begin{tabular}{l | l | l }
\multirow{3}{*}{\minitab[l]{ }} &
\multirow{3}{*}{\minitab[l]{kann man \\ lässt sich}} &
\multirow{3}{*}{\minitab[l]{eine Gleichung aufstellen \ldots \ }} \\
& & \\
& & \\
\end{tabular}
}
{
Bsp.: Kombiniert man die angegebenen Beziehungen, so lässt sich eine Gleichung aufstellen, welche die Anomalie recht gut beschreibt.
}

\mybox{
\begin{tabular}{l | l | l | l }
\multirow{7}{*}{\minitab[l]{Aus}} &
\multirow{7}{*}{\minitab[l]{geometrischen \\ trigonometrischen \\ mathematischen \\ den gleichen \\ bisherigen \\ theoretischen \\ oben angestellten}} &
\multirow{7}{*}{\minitab[l]{Gegebenheiten \ldots \  \\ Überlegungen \ldots \ }} &
\multirow{7}{*}{\minitab[l]{folgt \ldots \  \\ lässt sich \ldots \  ableiten.}} \\
& & & \\
& & & \\
& & & \\
& & & \\
& & & \\
& & & \\
\end{tabular}
}
{Bsp.: Aus trigonometrischen Überlegungen an der Abb. 2 folgt für den Winkel $\gamma$.}

\end{document}
