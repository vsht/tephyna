\documentclass[../tephyna.tex]{subfiles}
\begin{document}

\mybox{
\begin{tabular}{l}
Damit eröffnet sich die Möglichkeit \ldots \  \\
Das bisher Gesagte ist \ldots \  \\
Wie im vorhergehenden Abschnitt beschrieben, \ldots \  \\
Die bisher besprochenen \ldots \  \\
Theoretischen Überlegungen zufolge \ldots \  \\
So liegt die Vermutung nahe, \ldots \  \\
Deshalb ist \ldots \  vorzuziehen \\
\end{tabular}
}
{
Bsp.: Deshalb ist das Debye-Modell vorzuziehen.
}

\mybox{
\begin{tabular}{l | l | l }
\multirow{5}{*}{\minitab[l]{Damit}} &
\multirow{5}{*}{\minitab[l]{diese Aussage}} &
\multirow{5}{*}{\minitab[l]{einen Sinn ergibt \ldots \  \\ hält \ldots \  \\ zutrifft \ldots \  \\ gilt \ldots \  \\ ihre Gültigkeit behält \ldots \ }} \\
& & \\
& & \\
& & \\
& & \\
\end{tabular}
}
{Bsp.: Damit diese Aussage wirklich gilt, muss sie experimentell verifiziert werden.}

\mybox{
\begin{tabular}{l | l | l | l | l}
\multirow{5}{*}{\minitab[l]{Dass \ldots \ , hat}} &
\multirow{5}{*}{\minitab[l]{theoretische \\ praktische \\ verfahrenstechnische \\ entwicklungstechnische \\ gute}} &
\multirow{5}{*}{\minitab[l]{Gründe, die}} &
\multirow{5}{*}{\minitab[l]{später \\ nachfolgend \\ nicht näher \\ hier kurz \\ weiter unten}} &
\multirow{5}{*}{\minitab[l]{erläutert werden.}} \\
& & & & \\
& & & & \\
& & & & \\
& & & & \\
\end{tabular}
}
{
Bsp.: Dass man hauptsächlich bei Temperaturen unter 200K arbeitet, aht verfahrenstechnische Gründe, die hier nicht näher erläutert werden.
}

\mybox{
\begin{tabular}{ l | l | l}
\multirow{10}{*}{\minitab[l]{Am Rande \\ Der Vollständigkeit halber \\ Anschließend \\ Nebenbei \\ Zur Einleitung \\ Nur so viel \\ Beiläufig \\ an dieser Stelle \\ in diesem Zusammenhang}} &
\multirow{10}{*}{\minitab[l]{sei }} &
\multirow{10}{*}{\minitab[l]{erwähnt \ldots \  \\ darauf hingewiesen \ldots \  \\ nochmal betont \ldots \ }} \\
& & \\
& & \\
& & \\
& & \\
& & \\
& & \\
& & \\
& & \\
& & \\
\end{tabular}
}
{
Bsp.: An dieser Stelle sei erwähnt, dass die ART keine Theorie der Quantengravitation ist.
}

\mybox{
\begin{tabular}{l | l | l | l | l | l}
\multirow{5}{*}{\minitab[l]{Die}} &
\multirow{5}{*}{\minitab[l]{Überlegungen \\ Betrachtungen}} &
\multirow{5}{*}{\minitab[l]{des}} &
\multirow{5}{*}{\minitab[l]{vorausgehenden \\ vorigen \\ letzten \\ vorangegangenen \\ vorherigen}} &
\multirow{5}{*}{\minitab[l]{Abschnitts \\ Kapitels}} &
\multirow{5}{*}{\minitab[l]{zeig(t/en) \ldots \  }} \\
& & & & & \\
& & & & & \\
& & & & & \\
& & & & & \\
\end{tabular}
}
{
Bsp.: Die Überlegungen des letzten Kapitels zeigen, dass es für viele Probleme der Quantenfeldtheorie keine exakten analytischen Lösungen gibt.
}



\mybox{
\begin{tabular}{l | l | l | l }
\multirow{3}{*}{\minitab[l]{\ldots \  eignet sich somit}} &
\multirow{3}{*}{\minitab[l]{allein \\ ebenfalls \\ noch}} &
\multirow{3}{*}{\minitab[l]{nicht zur}} &
\multirow{3}{*}{\minitab[l]{Beschreibung \ldots \  \\ Bestimmung \ldots \ }} \\
& & & \\
& & & \\
\end{tabular}
}{
Bsp.: Die Spannung zwischen den Kondensatorplatten eignet sich somit allein nicht zur Beschreibung des zugehörigen elektrischen Feldes.
}

\mybox{
\begin{tabular}{l | l | l }
\multirow{4}{*}{\minitab[l]{Es liegt}} &
\multirow{4}{*}{\minitab[l]{daher \\ also \\ vielmehr \\ natürlich }} &
\multirow{4}{*}{\minitab[l]{nahe, zu vermuten, dass \ldots}} \\
& & \\
& & \\
& & \\
\end{tabular}
}
{
Bsp.: Es liegt natürlich nahe, zu vermuten, dass beide Teilchen miteinander wechselwirken können.
}

\mybox{
\begin{tabular}{l | l | l }
\multirow{8}{*}{\minitab[l]{\ldots \  unterscheidet sich somit}} &
\multirow{8}{*}{\minitab[l]{grundlegend \\ wesentlich \\ kaum \\ deutlich \\ grundsätzlich \\ signifikant \\ erheblich \\ tiefgreifend}} &
\multirow{8}{*}{\minitab[l]{von \ldots \ }} \\
& & \\
& & \\
& & \\
& & \\
& & \\
& & \\
& & \\
\end{tabular}
}
{
Bsp.: Das neue Phänomen unterscheidet sich somit grundsätzlich von dem Meissner-Effekt.
}

\end{document}
