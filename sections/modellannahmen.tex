\documentclass[../tephyna.tex]{subfiles}
\begin{document}

\mybox{
\begin{tabular}{l}
\ldots \  stellt eine Idealisierung dar. \\
\end{tabular}
}
{
Bsp.: Die Annahme, dass alle Moden mit der gleichen Frequenz schwingen, stellt eine Idealisierung dar.
}

\mybox{
\begin{tabular}{l | l | l | l | l}
\multirow{3}{*}{\minitab[l]{Mit}} &
\multirow{3}{*}{\minitab[l]{diesem einfachen \\ dem neuen \\ dem beschriebenen}} &
\multirow{3}{*}{\minitab[l]{Modell ist}} &
\multirow{3}{*}{\minitab[l]{unter anderem \\ leicht \\ außerdem}} &
\multirow{3}{*}{\minitab[l]{zu verstehen \ldots \  \\ zu erklären \ldots \ }} \\
& & & & \\
& & & & \\
\end{tabular}
}
{
Bsp.: Mit diesem einfachen Modell ist unter anderem zu verstehen, wie der Tunneleffekt zustande kommt.
}

\mydbox{
\begin{tabular}{l | l | l  l}
\multirow{5}{*}{\minitab[l]{\ldots \  lässt sich ein}} &
\multirow{5}{*}{\minitab[l]{mathematisches \\ sinnvolles \\realistisches \\ anschauliches \\ einfaches}} &
\multirow{5}{*}{\minitab[l]{Modell empirisch beobachtbarer physikalischer}} &
\multirow{5}{*}{\minitab[l]{\Return }} \\
& & & \\
& & & \\
& & & \\
& & & \\
\end{tabular}
}
{
\begin{tabular}{l | l | l}
\multirow{5}{*}{\minitab[l]} &
\multirow{5}{*}{\minitab[l]{Tatsachen \\ Sachverhalte \\ Phänomene}} &
\multirow{5}{*}{\minitab[l]{aufstellen. \\ formulieren. \\ entwickeln.}} \\
& & \\
& & \\
& & \\
& & \\
\end{tabular}
}
{
Bsp.: Damit lässt sich ein anschauliches Modell empirisch beobachtbarer physikalischer Tatsachen aufstellen.
}

\mybox{
\begin{tabular}{l | l | l | l }
\multirow{3}{*}{\minitab[l]{Für die weiteren Betrachtungen wird}} &
\multirow{3}{*}{\minitab[l]{ein \\ das}} &
\multirow{3}{*}{\minitab[l]{\ldots \ -modell}} &
\multirow{3}{*}{\minitab[l]{angenommen \\ verwendet \\ herangezogen}} \\
& & & \\
& & & \\
\end{tabular}
}
{
Bsp.: Für die weiteren Betrachtungen wird das Tröpfchenmodell verwendet.
}

\end{document}
