\documentclass[../tephyna.tex]{subfiles}
\begin{document}

\mybox{
\begin{tabular}{l}
\ldots \  anhand von \ldots \  Beispielen erläutern. \\
\ldots \  kommt eine große Bedeutung zu. \\
\end{tabular}
}
{
Bsp.: Die Bedeutung der Störungsrechnung für die theoretische Physik lässt sich anhand von drei folgenden Beispielen erläutern.
}

\mydbox{
\begin{tabular}{l | l | l | l | l }
\multirow{4}{*}{\minitab[l]{Trotz \\ Ungeachtet}} &
\multirow{4}{*}{\minitab[l]{der inhaltlichen}} &
\multirow{4}{*}{\minitab[l]{Verschiedenheit \\ Heterogenität \\ Unterschiedlichkeit}} &
\multirow{4}{*}{\minitab[l]{dieser}} &
\multirow{4}{*}{\minitab[l]{Beispiele \\ Erscheinungen \quad \Return \\ Phänomene}} \\
& & & & \\
& & & & \\
& & & & \\
\end{tabular}
}
{
\begin{tabular}{l | l | l | l }
\multirow{4}{*}{\minitab[l]{ }} &
\multirow{4}{*}{\minitab[l]{handelt es sich \\ haben sie \\ unterscheiden sie sich \\ stellen sie}} &
\multirow{4}{*}{\minitab[l]{mathematisch \\ physikalisch}} &
\multirow{4}{*}{\minitab[l]{gesehen, \ldots \ }} \\
& & & \\
& & & \\
& & & \\
\end{tabular}
}
{
Bsp.: Trotz der inhaltlichen Verschiedenheit dieser Erscheinungen, handelt es sich physikalisch gesehen um dasselbe Phänomen - die Interferenz.
}

\mybox{
\begin{tabular}{l | l | l | l }
\multirow{9}{*}{\minitab[l]{Ein}} &
\multirow{9}{*}{\minitab[l]{weiteres \\ anderes \\ beonders \\ technisch \\ physikalisch \\ überaus \\ außerordentlich \\ sehr \\ äußerst}} &
\multirow{9}{*}{\minitab[l]{wichtiges \\ eindrucksvolles}} &
\multirow{9}{*}{\minitab[l]{Beispiel für \ldots \  ist \ldots \ }} \\
& & & \\
& & & \\
& & & \\
& & & \\
& & & \\
& & & \\
& & & \\
& & & \\
\end{tabular}
}
{
Bsp.: Ein besonders eindrucksvolles Beispiel für die Welleneigenschaften des Lichts ist Beugung am Einfachspalt.
}

\mybox{
\begin{tabular}{l | l | l }
\multirow{5}{*}{\minitab[l]{Ein(e) solche(r/s)}} &
\multirow{5}{*}{\minitab[l]{Anordnung \\ Gerät \\ Anlage \\ Versuchsaufbau \\ Messaufbau}} &
\multirow{5}{*}{\minitab[l]{lässt sich zu \ldots \  verwenden. \\ dient zu \ldots \\ kommt immer dann in Betracht, wenn \ldots \ }} \\
& & \\
& & \\
& & \\
& & \\
\end{tabular}
}
{
Bsp.: Ein solcher Messaufbau kommt immer dann in Betracht, wenn man an einer äußerst präzisen Frequenzbestimmung interessiert ist.
}

\mybox{
\begin{tabular}{l | l }
\multirow{5}{*}{\minitab[l]{Die Anwendungsgebiete \ldots \  sind}} &
\multirow{5}{*}{\minitab[l]{vielfältig \ldots \\ nahezu unbegrenzt \ldots  \\ (sehr) weitreichend \ldots \\ typisch \ldots  \\ zum Beispiel \ldots }} \\
 & \\
 & \\
 & \\
 & \\
\end{tabular}
}
{
Bsp.: Die Anwendungsgebiete der Röntgenstrahlung sind sehr vielfältig, wobei medizinische Anwendungen eine besonders wichtige Rolle spielen.
}

\subsection{Fragestellungen}
\mybox{
\begin{tabular}{ l | l}
\multirow{5}{*}{\minitab[l]{Dabei Stellt sich allerdings die Frage, \\ Die Frage,}} &
\multirow{5}{*}{\minitab[l]{ob \ldots \  \\ was \ldots \  \\ warum \ldots \  \\ wie \ldots \  \\ wann \ldots \  }} \\
& \\
& \\
& \\
& \\
\end{tabular}
}
{
Bsp.: Die Frage, ob die Existenz solcher Teilchen grundsätzlich ausgeschlossen werden kann, lässt sich nicht eindeutig beantworten.
}

\end{document}
