\documentclass[../tephyna.tex]{subfiles}
\begin{document}

\subsection{Einleitungen}

\mybox{
\begin{tabular}{l | l | l }
\multirow{4}{*}{\minitab[l]{Diese(r/s) }} &
\multirow{4}{*}{\minitab[l]{beeindruckende \\ wirkungsvolle \\ effektvolle \\ eindrucksvolle }} &
\multirow{4}{*}{\minitab[l]{Versuch \ldots \  \\ Experiment \ldots \ }} \\
& & \\
& & \\
& & \\
\end{tabular}
}
{
Bsp.: Dieser wirkungsvolle Versuch wurde zum ersten Mal von Müller vorgeschlagen, konnte aber erst 30 Jahre später durchgeführt werden.
}

\mydbox{
\begin{tabular}{l | l | l | l }
\multirow{4}{*}{\minitab[l]{Zur}} &
\multirow{4}{*}{\minitab[l]{experimentellen Überprüfung \\ Veranschaulichung}} &
\multirow{4}{*}{\minitab[l]{obiger \\ oben angeführter \\ oben genannter \\ oben dargestellter}} &
\multirow{4}{*}{\minitab[l]{Überlegungen \\ Darlegungen \quad \Return \\ Vermutungen}} \\
 & & & \\
 & & & \\
 & & & \\
\end{tabular}
}
{
\begin{tabular}{l | l | l }
\multirow{4}{*}{\minitab[l]{ }} &
\multirow{4}{*}{\minitab[l]{dient \\ eignet sich}} &
\multirow{4}{*}{\minitab[l]{folgender Versuch. \\ folgendes Gedankenexperiment.}} \\
& & \\
& & \\
& & \\
\end{tabular}
}
{
Bsp.: Zur experimenteller Überprüfung oben angeführter Überlegungen eignet sich folgender Versuch.
}

\mydbox{
\begin{tabular}{l | l | l | l }
\multirow{5}{*}{\minitab[l]{Ein}} &
\multirow{5}{*}{\minitab[l]{einfaches \\ geniales \\ weiteres \\ faszinierendes \\ sorgfältig durchdachtes}} &
\multirow{5}{*}{\minitab[l]{Experiment,}} &
\multirow{5}{*}{\minitab[l]{das \quad \quad \quad \Return \\ welches }} \\
& & \\
& & \\
& & \\
& & \\
\end{tabular}
}
{
\begin{tabular}{l|l | l | l}
\multirow{5}{*}{\minitab[l]{}} &
\multirow{5}{*}{\minitab[l]{\ldots \  zu messen \\ die Messung  \ldots \  \\ die messtechnische \Return \\ Bestimmung  \ldots \ }} &
\multirow{5}{*}{\minitab[l]{gestattet, \\ erlaubt,}} &
\multirow{5}{*}{\minitab[l]{ist \ldots \  \\ führte(n) zum ersten Mal \ldots \   \Return \\ im Jahr \ldots \  durch. \\ wurde erstmals von \ldots \  im Jahre \Return \\ durchgeführt.}} \\
& & \\
& & \\
& & \\
& & \\
\end{tabular}
}
{
Bsp.: Ein sorgfältig durchdachtes Experiment, welches die Messung der Gravitationskonstante gestattet, führte zum ersten Mal Cavendish im Jahr 1797 durch.
}

\mybox{
\begin{tabular}{l | l}
\multirow{2}{*}{\minitab[l]{Der (experimentelle) Nachweis \ldots \ }} &
\multirow{2}{*}{\minitab[l]{kann durch folgenden Versuch erbracht werden. \\ erfolgt mit Hilfe des folgenden Versuchs.}} \\
 & \\
\end{tabular}
}
{
Bsp.: Der experimentelle Nachweis elektromagnetischer Wellen kann durch folgenden Versuch erbracht werden.
}

\mybox{
\begin{tabular}{l | l | l  }
\multirow{4}{*}{\minitab[l]{In der Folgezeit wurde(n)}} &
\multirow{4}{*}{\minitab[l]{noch weitere \ldots \  \\ \ldots \  weitere \ldots \  \\ zahlreiche neue \ldots \  \\ eine Reihe von \ldots \  }} &
\multirow{4}{*}{\minitab[l]{entdeckt}} \\
& & \\
& & \\
& & \\
\end{tabular}
}
{
Bsp.: In der Folgezeit wurden 2 weitere Teilchen entdeckt.
}

\mybox{
\begin{tabular}{l | l | l }
\multirow{3}{*}{\minitab[l]{Die von verschiedenen Forschern}} &
\multirow{3}{*}{\minitab[l]{beobachteten \\ durchgeführten \\ ermittelten}} &
\multirow{3}{*}{\minitab[l]{\ldots \ }} \\
& & \\
& & \\
\end{tabular}
}
{
Bsp.: Die von verschiedenen Forschern ermittelten Werte unterscheiden sich um weniger als $1\,\sigma$.
}

\subsection{Erklärungen}

\mybox{
\begin{tabular}{l}
Damit erreicht man, dass \ldots \  \\
Zu diesem Zweck \ldots \  \\
Durch die Wirkung \ldots \  lässt sich \ldots \  im Fall des oben dargestellten Versuchs erklären \\
Die benutzten Geräte \ldots \  \\
Zur experimentellen Untersuchung \ldots \  \\
\end{tabular}
}
{
Bsp.: Die benutzten Geräte werden vor der Versuchsdurchführung sorgfältig kalibriert.
}

\mybox{
\begin{tabular}{ l | l }
\multirow{9}{*}{\minitab[l]{Weitere Messungen}} &
\multirow{9}{*}{\minitab[l]{zeigen \ldots \  \\ belegen \ldots \  \\ bestätigen \ldots \  \\ ermöglichen \ldots \  \\ werden \ldots \  durchgeführt. \\ werden benötigt, \ldots \  \\ werden es möglich machen, \ldots \  \\ werden \ldots \  vorgenommen. \\ werden \ldots \  erfolgen.}} \\
& \\
& \\
& \\
& \\
& \\
& \\
& \\
& \\
\end{tabular}
}
{
Bsp.: Weitere Messungen werden an den Proben aus Aluminium vorgenommen.
}

\mydbox{
\begin{tabular}{l | l }
\multirow{4}{*}{\minitab[l]{Der Zusammenhang zwischen \ldots \  und \ldots \  lässt sich mit der folgenden }} &
\multirow{4}{*}{\minitab[l]{Anordnung \\ Messanordnung \quad \quad  \Return \\ Versuchsanordnung}} \\
 & \\
 & \\
 & \\
\end{tabular}
}
{
\begin{tabular}{ l | l }
\multirow{4}{*}{\minitab[l]{ }} &
\multirow{4}{*}{\minitab[l]{untersuchen. \\ bestimmen. \\ überprüfen. \\ darstellen.}} \\
 & \\
  & \\
 & \\
\end{tabular}
}
{
Bsp.: Der Zusammenhang zwischen der Teilchengeschwindigkeit und Linienbreite lässt sich mit der folgenden Versuchsanordnung bestimmen.
}

\mybox{
\begin{tabular}{l | l}
\multirow{4}{*}{\minitab[l]{Um den zeitlichen Verlauf \ldots \ }} &
\multirow{4}{*}{\minitab[l]{verfolgen zu können, \ldots \  \\ untersuchen zu können, \ldots \  \\ festzuhalten, \ldots \  \\ darzustellen, \ldots \  }} \\
 & \\
& \\
& \\
\end{tabular}
}
{
Bsp.: Um den zeitlichen Verlauf dieses Prozesses untersuchen zu können, braucht man abstimmbare Laser, die höchste Leistung bei kleinsten Wellenlängen liefern.
}

\mydbox{
\begin{tabular}{l | l | l | l | l | l }
\multirow{2}{*}{\minitab[l]{Die Messung \ldots \  ist}} &
\multirow{2}{*}{\minitab[l]{(relativ) \\ (sehr)}} &
\multirow{2}{*}{\minitab[l]{kompliziert \\ aufwändig}} &
\multirow{2}{*}{\minitab[l]{und}} &
\multirow{2}{*}{\minitab[l]{erfordert \\ verlangt}} &
\multirow{2}{*}{\minitab[l]{daher \quad \Return}} \\
 & & & & \\
\end{tabular}
}
{
\begin{tabular}{l | l }
\multirow{10}{*}{\minitab[l]{}} &
\multirow{10}{*}{\minitab[l]{viel Erfahrung. \\ große Sorgfalt. \\  \multirow{4}{*}{\minitab[l|]{ tiefe \\ spezielle \\ fundierte \\ besondere}} \multirow{4}{*}{\minitab[l]{ Fachkenntnisse.}} \\ \\ \\ \\  \multirow{4}{*}{\minitab[l]{einen hohen}} \multirow{4}{*}{\minitab[|l|]{ experimentellen \\ finanziellen \\ zeitlichen \\ personellen}} \multirow{4}{*}{\minitab[l]{Aufwand.}} \\ \\ \\ \\  }} \\
 & \\
 & \\
 & \\
 & \\
 & \\
 & \\
 & \\
 & \\
 & \\
\end{tabular}
}
{
Bsp.: Die Messung des Korrelationsparameters ist relativ kompliziert und erfordert daher große Sorgfalt und viel Erfahrung.
}

\mybox{
\begin{tabular}{l | l | l  }
\multirow{3}{*}{\minitab[l]{In der Praxis \\ Im Laboralltag \\ Bei solchen Experimenten}} &
\multirow{3}{*}{\minitab[l]{ist es}} &
\multirow{3}{*}{\minitab[l]{üblich \ldots \ \\ gebräuchlich \ldots \ \\ wichtig \ldots \ }} \\
 & & \\
 & & \\
\end{tabular}
}
{
Bsp.: In der Praxis ist es üblich, statt der Wellenlängen die Wellenzahlen zu verwenden.
}

\mybox{
\begin{tabular}{l | l | l }
\multirow{3}{*}{\minitab[l]{Bei mehrmaliger Wiederholung}} &
\multirow{3}{*}{\minitab[l]{des Versuchs \\ des Experiments \\ der Messung}} &
\multirow{3}{*}{\minitab[l]{stellt man fest, \ldots \ }} \\
& & \\
& & \\
\end{tabular}
}
{
Bsp.: Bei mehrmaliger Wiederholung der Messung stellt man fest, dass die Messwerte nicht der Gaußverteilung gehorchen.
}

\mybox{
\begin{tabular}{l | l | l | l }
\multirow{5}{*}{\minitab[l]{\ldots \  muss bei Präzisionsmessungen}} &
\multirow{5}{*}{\minitab[l]{unbedingt \\ entsprechend \\ um jeden Preis \\ möglichst \\ durch \ldots \ }} &
\multirow{5}{*}{\minitab[l]{berücksichtigt \\ beachtet \\ vermieden \\ reduziert \\ verhindert}} &
\multirow{5}{*}{\minitab[l]{werden.}} \\
& & & \\
& & & \\
& & & \\
& & & \\
\end{tabular}
}
{
Bsp.: Eine starke Erwärmung des Messobjekts muss bei Präzisionsmessungen unbedingt verhindert werden.
}

\subsection{Interpretationen}

\mybox{
\begin{tabular}{l}
In Übereinstimmung mit dem Experiment\\
Die Übereinstimmung mit der Theorie \\
Dabei ist anzunehmen, dass \ldots \  \\
Ihnen allen ist gemeinsam, \ldots \  \\
Zur Deutung \ldots \
\end{tabular}
}
{
Bsp.: Zur Deutung dieses Phänomens bedarf es einer neuen Theorie jenseits des Standardmodells.
}

\mybox{
\begin{tabular}{l | l | l | l }
\multirow{7}{*}{\minitab[l]{Mit}} &
\multirow{7}{*}{\minitab[l]{einer \\ der}} &
\multirow{7}{*}{\minitab[l]{einfachen \\ speziellen \\ modifizierten \\ teureren \\ besonderen \\ entsprechenden \\ geschickten}} &
\multirow{7}{*}{\minitab[l]{Versuchsanordnung kann man zeigen, \ldots \ }} \\
& & & \\
& & & \\
& & & \\
& & & \\
& & & \\
& & & \\
\end{tabular}
}
{
Bsp.: Mit der entsprechenden Versuchsanordnung kann man zeigen, dass die Energie der Elektronen von der Intensität des einfallenden Lichts unabhängig ist.
}

\mydbox{
\begin{tabular}{l | l | l }
\multirow{3}{*}{\minitab[l]{Stimmt der}} &
\multirow{3}{*}{\minitab[l]{berechnete \\ eingestellte \\ angegebene}} &
\multirow{3}{*}{\minitab[l]{Wert im Rahmen der Messgenauigkeit mit dem \Return}} \\
 & & \\
 & & \\
\end{tabular}
}
{
\begin{tabular}{l | l | l}
\multirow{5}{*}{\minitab[l]{ }} &
\multirow{5}{*}{\minitab[l]{experimentell ermittelten \\ gemessenen \\ angezeigten \\ geschätzten \\ vermuteten}} &
\multirow{5}{*}{\minitab[l]{überein, so \ldots \ }} \\
& & \\
& & \\
& & \\
& & \\
\end{tabular}
}
{
Bsp.: Stimmt der berechnete Wert im Rahmen der Messgenauigkeit mit dem experimentell ermittelten überein, so kann die Vorhersage der Theorie offenbar verifiziert werden.
}

\mybox{
\begin{tabular}{l | l | l | l }
\multirow{7}{*}{\minitab[l]{Bestimmung \ldots \  unter Verwendung}} &
\multirow{7}{*}{\minitab[l]{de(r/s) }} &
\multirow{7}{*}{\minitab[l]{bisher entwickelten \\ bewährten \\ erweiterten \\ modernen \\ vereinfachten \\ modifizierten \\ weiterentwickelten}} &
\multirow{7}{*}{\minitab[l]{Theorie \ldots \ \\ Modells \ldots \ \\ Näherung \ldots \ }} \\
& & & \\
& & & \\
& & & \\
& & & \\
& & & \\
& & & \\
\end{tabular}
}
{
Bsp.: Bestimmung der Ionisationsenergie unter Verwendung der vereinfachten Theorie liefert hier erstaunlich gute Werte.
}

\mybox{
\begin{tabular}{l | l | l }
\multirow{8}{*}{\minitab[l]{Die Interpretation der \\}} &
\multirow{8}{*}{\minitab[l]{experimentellen \Return \\ Befunde \\ Versuchsergebnisse \\ erhaltenen Ergebnisse \\ berechneten Werte \\ experimentellen \Return \\ Resultate \\ Messwerte }} &
\multirow{8}{*}{\minitab[l]{
\multirow{2}{*}{\minitab[l|]{ steht \\ stehen} \minitab[l]{in guter Übereinstimmung mit \\ im Einklang mit}}
 \\ \\ \multirow{2}{*}{\minitab[l|]{ erweist \\ erweisen} sich als \ldots \  } \\ \\ \multirow{2}{*}{\minitab[l|]{ lässt \\ lassen} folgende Vermutungen zu \ldots \  } }} \\
& & \\
& & \\
& & \\
& & \\
& & \\
& & \\
& & \\
\end{tabular}
}
{
Bsp.: Die ermittelten Ergebnisse stehen im Einklang mit den Vorhersagen des Standardmodells.
}

\mybox{
\begin{tabular}{ l | l }
\multirow{4}{*}{\minitab[l]{Damit \\ So \\ Auf diese Weise \\ Dadurch}} &
\multirow{4}{*}{\minitab[l]{erhält man Aufschluss über \ldots \ }} \\
& \\
& \\
& \\
\end{tabular}

}
{
Bsp.: Auf diese Weise erhält man Aufschluss über den Energieabstand der beiden Niveaus.
}

\mydbox{
\begin{tabular}{l | l | l}
\multirow{3}{*}{\minitab[l]{Dieser Versuch \\ Dieses Experiment}} &
\multirow{3}{*}{\minitab[l]{dient dazu, \\ hilft dabei, \\ eignet sich dazu, }} &
\multirow{3}{*}{\minitab[l]{Zusammenhänge zwischen Theorie und \Return }} \\
 & & \\
 & & \\
\end{tabular}
}
{
\begin{tabular}{l | l | l}
\multirow{5}{*}{\minitab[l]{ }} &
\multirow{5}{*}{\minitab[l]{Praxis \\ Wirklichkeit \\ Experiment \\ empirischer Forschung \\ empirischen Beobachtungen}} &
\multirow{5}{*}{\minitab[l]{zu ermitteln. \\ zu erforschen. \\ herzustellen. \\ zu veranschaulichen.}} \\
& & \\
& & \\
& & \\
& & \\
\end{tabular}
}
{
Bsp.: Dieser Versuch dient dazu, Zusammenhänge zwischen Theorie und Experiment zu veranschaulichen.
}

\mybox{
\begin{tabular}{l | l | l }
\multirow{10}{*}{\minitab[l]{Dieser Versuch \\ Dieses Experiment}} &
\multirow{10}{*}{\minitab[l]{dient als \\ gilt als }} &
\multirow{10}{*}{\minitab[l]{Vorbereitung \ldots \  \\ Nachweis \ldots \  \\ Alternative \ldots \ \\ Einstieg \ldots \ \\ Einführung \ldots \ \\ Beweis \ldots \ \\ Demonstration \ldots \  \\ Motivation \ldots \ \\ Beispiel \ldots \  \\ Simulation \ldots \ }} \\
& & \\
& & \\
& & \\
& & \\
& & \\
& & \\
& & \\
& & \\
& & \\
\end{tabular}
}
{
Bsp.: Dieser Versuch gilt als Einführung in die moderne Laserspektroskopie.
}

\mybox{
\begin{tabular}{ l | l }
\multirow{4}{*}{\minitab[l]{Dieser Versuch \\ Dieses Experiment}} &
\multirow{4}{*}{\minitab[l]{lässt folgende Vermutungen zu \ldots \  \\ ergibt, dass \ldots \  \\ ermöglicht ein tieferes Verständnis \ldots \  \\ verdeutlicht \ldots \ }} \\
& \\
& \\
& \\
\end{tabular}
}
{
Bsp.: Dieses Experiment ergibt, dass sich das Modell eines Lorentz-Oszillators hier hervorragend anwenden lässt.
}

\mydbox{
\begin{tabular}{l | l | l | l | l | l }
\multirow{4}{*}{\minitab[l]{Die(se) }} &
\multirow{4}{*}{\minitab[l]{Ergebnis(se) \\ Vermutung(en) \\ Aussage(n) \\ Messwert(e)}} &
\multirow{4}{*}{\minitab[l]{müssen \\ können}} &
\multirow{4}{*}{\minitab[l]{durch}} &
\multirow{4}{*}{\minitab[l]{weitere \\ nachfolgende \\ künftige \\ spätere}} &
\multirow{4}{*}{\minitab[l]{Versuche \quad \Return}} \\
 & & & & & \\
 & & & & & \\
 & & & & & \\
\end{tabular}
}
{
\begin{tabular}{l | l | l | l }
\multirow{4}{*}{\minitab[l]{  }} &
\multirow{4}{*}{\minitab[l]{nicht \\ hervorragend \\ nur unzureichend }} &
\multirow{4}{*}{\minitab[l]{bestätigt \\ verifiziert \\ präzisiert \\ reproduziert }} &
\multirow{4}{*}{\minitab[l]{werden.}} \\
& & & \\
& & & \\
& & & \\
\end{tabular}
}
{
Bsp.: Dieses Ergebnis konnte allerdings durch weitere Versuche nicht bestätigt werden.
}

\mybox{
\begin{tabular}{l | l | l | l | l}
\multirow{5}{*}{\minitab[l]{Schlüsse, \\ Folgerungen, \\ Rückschlüsse,}} &
\multirow{5}{*}{\minitab[l]{die man als}} &
\multirow{5}{*}{\minitab[l]{Physiker \\ Experimentator \\ Theoretiker}} &
\multirow{5}{*}{\minitab[l]{daraus zu \Return \\ ziehen hat,}} &
\multirow{5}{*}{\minitab[l]{sind \ldots \  \\ lassen sich \ldots \  \\ beschränken \Return \\ sich auf \ldots \ }} \\
& & & & \\
& & & & \\
& & & & \\
& & & & \\
\end{tabular}
}
{
Bsp.: Schlüsse die man als Physiker aus dem Michelson-Morley-Experiment zu ziehen hat, lassen sich in einem einzigen kurzen Satz zusammenfassen: Es gibt keinen Weltäther!
}


\end{document}
