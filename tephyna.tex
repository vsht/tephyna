\documentclass[10pt,a4paper]{scrartcl}
\usepackage{scrhack}
\usepackage[a4paper, includefoot, top=2cm, bottom=2 cm, left= 1.5cm, right=1.3cm]{geometry}
\usepackage{newcent} % New century schoolbook font
\usepackage{subfiles} % splitting document into parts in a more convenient way
\usepackage[T1]{fontenc} %encoding
\usepackage[utf8]{inputenc} % encoding
\usepackage[ngerman]{babel} % spelling
\usepackage{hyperref} % references
\usepackage[onehalfspacing]{setspace} %spacing
\usepackage{multirow} % special tables
\usepackage{fancyhdr} % header and footers
\usepackage{tcolorbox} % boxes around text
\usepackage{keystroke} % carriare return symbol
\usepackage{xcolor} % colors
\usepackage{titlesec} % to start each section with new page
\setlength{\parindent}{0ex}
\definecolor{lightgray}{gray}{0.93}
\newcommand{\sectionbreak}{\clearpage}
\newcommand{\minitab}[2][l]{\begin{tabular}{#1}#2\end{tabular}}
\newcommand{\mybox}[2]{\begin{tcolorbox}[colback=white]\colorbox{lightgray}{#1}\\ \\ \texttt{#2} \end{tcolorbox}}
\newcommand{\mydbox}[3]{\begin{tcolorbox}[colback=white]\colorbox{lightgray}{#1} \vskip 2mm \colorbox{lightgray}{#2}\\ \\ \texttt{#3} \end{tcolorbox}}

\begin{document}

\subfile{sections/vorwort.tex}
\newpage

\pagestyle{fancy}
\renewcommand{\headrulewidth}{0.4pt}
\renewcommand{\footrulewidth}{0.4pt}
\fancyhead[L]{TePhyNa - Textbausteine für Physiker und andere Naturwissenschaftler}
\fancyhead[C]{~}
\fancyhead[R]{\rightmark}
\fancyfoot[OL]{Seite \thepage}
\fancyfoot[OC]{~}
\fancyfoot[OR]{Copyright (C)  2010-2014 Vladyslav Shtabovenko, CC-BY-SA 4.0}

\tableofcontents

\newpage

\section{Einleitungen}
\subfile{sections/einleitung.tex}
\section{Bilder, Tabellen, Diagramme, Graphen}

\subfile{sections/bilder.tex}
\section{Versuche}
\subfile{sections/versuche.tex}
\section{Anwendungen, Beispiele}
\subfile{sections/anwendungen.tex}
\section{Argumente und Erklärungen}
\subfile{sections/argumente.tex}
\section{Überleitungen}
\subfile{sections/ueberleitungen.tex}
\section{Modellannahmen}
\subfile{sections/modellannahmen.tex}
\section{Definitionen}
\subfile{sections/definitionen.tex}
\section{Berechnungen}
\subfile{sections/berechnungen.tex}

\end{document}
